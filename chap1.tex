\chapter{Introduction}
\label{chap:introduction}

% Introduction should provide appropriate context and background for your research, such as the recent trend and importance of the technology development related to your thesis.

Non terrestrial network (NTN) has become a promising technique in 6G network. It provides network connectivity to area that traditional network platform cannot reach. For instance, forests, oceans, and deserts. Various platforms are used to provide network services in NTN, such as GEO, MEO, LEO satellites, UAVs, and drones. Among these platforms, LEO is the most actively discussed since it provides global coverage, low latency, and high throughput. 

However, to acheive LEO satellite network service, there are some key challanges that need to be resolved. One of the challanges is the unavoidable frequent handover. The high speed of LEO satellite forces user equipments (UEs) on the ground to switch the serving satellites frequently~\cite{38821}. 3GPP has discussed some solutions to deal with this issue. By using quasi-earth-fixed cell and satellite switch with re-synchronization, the frequent handovers are avoided and the signalling overhead is reduced~\cite{38300}. Also, with the help of the ephemeris data of satellies and the position information of UEs, the matching between satellites and UEs has been largely improved~\cite{38331}. Nontheless, it still has some problems during the random access procedure. 